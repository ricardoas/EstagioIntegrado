\documentclass{article}

\usepackage[brazil]{babel}
\usepackage[latin1]{inputenc}
\usepackage{epsfig}
\usepackage{times}
\usepackage{amsfonts,amssymb,latexsym,eucal,amsmath,theorem,amscd}
\usepackage{graphicx}
\usepackage[T1]{fontenc}
\setlength{\textheight}{247mm}
\setlength{\textwidth}{155mm}
\setlength{\topmargin}{-0.4mm}
\setlength{\oddsidemargin}{5.4mm}
\setlength{\voffset}{-13mm}
\usepackage{colortbl}
\usepackage{abntcite}
\begin{document}

% 
% \bibliographystyle{plain}

\begin{titlepage}
    \begin{center}
        {\large
        UNIVERSIDADE FEDERAL DE CAMPINA GRANDE - UFCG \\
        CENTRO DE ENGENHARIA EL�TRICA E INFORM�TICA - CEEI \\
     	DEPARTAMENTO DE SISTEMAS E COMPUTA��O - DSC \\
	}
        \vspace{3cm}
        \vspace{5cm}
        PLANO DE EST�GIO\\
        Avalia��o qualitativa das principais tecnologias atuais\\ 
        de grades computacionais oportunistas\\
        \vspace{3cm}
		Ricardo Ara�jo Santos \\
        Curso de Bacharelado em Ci�ncia da Computa��o \\
	\vspace{7cm}        
	Campina Grande, Mar�o de 2009
    \end{center}
\end{titlepage}

\tableofcontents

\newpage
\section{Informa��es Pessoais}
\label{info}

\textbf{Nome:} Ricardo Ara�jo Santos \newline
\textbf{Matr��cula:} 20511120 \newline
\textbf{Endere�o Residencial:} Rua Oleg�rio de Azevedo, 268 / ap 01 - S�o Jos� - Campina Grande - PB \newline
\textbf{Endere�o Profissional:} Av. Apr�gio Veloso, 882  Bodocong�, Bloco CO \\
	58109-970, Campina Grande, PB \\
\textbf{Fone}: +55 (83) 3310 1640 , (83) 8875 7392\\
\textbf{Fax}: +55 (83) 3310 1498 \\
\textbf{E-mail}: ricardo [AT] lsd.ufcg.edu.br \\ \\


\newpage
\chapter{Ambiente de Est�gio}
\label{ambiente}

O est�gio foi desenvolvido no Laborat�rio de Sistemas Distribu�dos (LSD) do Departamento de Sistemas e Computa��o (DSC) da Universidade Federal de Campina Grande.

O laborat�rio foi criado em 1996, como forma de aglutinar pesquisadores e alunos do DSC e de outros departamentos em torno de projetos na �rea de Sistemas Distribu�dos. A\-tu\-al\-men\-te o LSD � coordenado pelo professor Francisco Brasileiro. As pesquisas do LSD est�o concentradas em Grades Computacionais, Sistemas \textit{Peer-to-Peer}, \textit{Cloud Computing}, Toler�ncia a Falhas, Desenvolvimento de Software Distribu�do e Aplica��es Industriais.

\section{Estrutura F�sica}
\label{estrutura}

O LSD est� instalado num pr�dio com $550m^2$ de �rea e conta com a colabora��o de dezenas de alunos desenvolvendo trabalhos de doutorado, mestrado e inicia��o cient�fica, al�m de v�rios pesquisadores. Os diferentes projetos em execu��o est�o distribu�dos em 8 (oito) salas climatizadas, cada uma contanto com quadro branco, pinc�is e um acervo bibliogr�fico de diversos temas em computa��o e engenharia. Cada pessoa possui um posto de trabalho individual, com m�quinas conectadas � Internet via POP-PB da RNP\nomenclature{RNP}{Rede Nacional de Pesquisa}.

\vspace{1cm}
\noindent
\textbf{Endere�o}: Universidade Federal de Campina Grande\newline
Departamento de Sistemas e Computa��o\newline
Laborat�rio de Sistemas Distribu�dos\newline
Av. Apr��gio Veloso, 882 - Bloco CO\newline
Bodocong�, CEP 58109-970 Campina Grande - PB, Brasil\newline
Fone: +55 83 3310 1365\newline
Fax: +55 83 3310 1498

\section{Rucursos Utilizados}

\subsection{Hardware}

Foram disponibilizados quatro computadores de configura��es distintas. 

\subsection{Software}

Os recursos de software utilizados foram os seguintes:

\begin{itemize}
  \item Eclipse: Como ambiente de desenvolvimento da aplica��o Java usada para testar as grades computacionais. Al�m disso, o Eclipse em conjunto com o plugin Texlipse foi usado como editor de texto \LaTeX;
  \item VServer: Como ambiente de m�quinas virtuais.
\end{itemize}


\newpage
\section{Supervis�o (Acad�mica e T�cnica)}
\label{supervisao}

A supervis�o acad�mica ser� efetuada pela professora Raquel Vigolvino Lopes, pesquisadora do Laborat�rio de Sistemas Distribu�dos (LSD) e professora do DSC/UFCG. A supervis�o t�cnica ser� efetuada pelo aluno de mestrado Marcus Williams Aquino de Carvalho.
% Manel?
\vspace{1cm}

\noindent
\textbf{Dados do supervisor acad�mico} \\
\textbf{Nome}: Raquel Vigolvino Lopes \\
\textbf{Endere�o Profissional}: Av. Apr�gio Veloso, 882  Bodocong�, Bloco CO \\
	58109-970, Campina Grande, PB \\
\textbf{Fone}: +55 (83) 3310 1643 \\
\textbf{Fax}: +55 (83) 3310 1498 \\
\textbf{E-mail}: raquel [AT] dsc.ufcg.edu.br \\ \\

\noindent
\textbf{Dados do supervisor t�cnico} \\
\textbf{Nome}: Marcus Williams Aquino de Carvalho \\
\textbf{Endere�o Profissional}: Av. Apr�gio Veloso, 882  Bodocong�, Bloco CO \\
	58109-970, Campina Grande, PB \\
\textbf{Fone}: +55 (83) 3310 1640 \\
\textbf{E-mail}: marcuswac [AT] lsd.ufcg.edu.br \\ \\



\newpage
\section{Resumo do Problema}
\label{problema}

A pesquisa cient�fica em algumas �reas tem demandado cada vez mais poder computacional, seja para a realiza��o de simula��es como para a execu��o de experimentos. Uma sa�da natural � a aquisi��o de supercomputadores ou m�quinas dedicadas (clusters) exclusivamente � computa��o paralela, o que nem sempre � poss�vel dado seu elevado custo. As grades computacionais \cite{berman, kesselman} surgiram com a id�ia de resolver tais problemas computacionais de maneira eficiente e com baixo custo, atrav�s do compartilhamento de recursos.

Uma forma comum de compartilhamento de recursos faz uso de poder computacional ocioso, formando o que se chama no contexto desse trabalho de grades computacionais oportunistas. Exemplos de grades oportunistas s�o: Condor~\cite{condor}, BOINC~\cite{boinc}, XtremWeb~\cite{xtremweb} e OurGrid~\cite{ourgrid}.

Muito tem sido publicado em confer�ncias e jornais sobre essas ferramentas. No entanto, n�o s�o conhecidos estudos emp�ricos que visem uma caracteriza��o do uso das ferramentas disponibilizadas, impossibilitando identificar o estado-da-pr�tica desta �rea. Acredita-se que um estudo pr�tico, em contrapartida aos estudos te�ricos conhecidos, � de grande import�ncia tanto para identifica��o de lacunas quanto para sele��o do melhor servi�o em diferentes contextos.

# transformar aquelas anota��es (nas folhas da conversa LSD) em texto.


\section{Objetivos}
\label{objetivos}

O principal objetivo desse trabalho � comparar de forma qualitativa o uso das principais tecnologias de grades desktops ou oportunistas da atualidade.

Como objetivos espec�ficos espera-se:

\begin{itemize}
  \item Realizar um levantamento bibliogr�fico sobre as grades desktop ou oportunistas mais usadas atualmente;
  \item Levantar m�tricas que possam ser usadas nesse estudo comparativo qualitativo;
  \item Comparar qualitativamente, com base nas m�tricas definidas, as grades identificadas no in�cio do est�gio. 
\end{itemize}


\newpage
\section{Metodologia}
\label{metodologia}

Uma abordagem qualitativa � mais adequada uma vez que existem caracter�sticas pr�prias dessas grades que n�o podem ser medidas, somente observadas. Inicialmente, pretende-se construir uma base de informa��es sobre as ferramentas a serem estudadas, tais como artigos e documenta��o das ferramentas. No entanto, a base de nossa metodologia � a pr�tica, de forma que, ser�o determinadas algumas m�tricas com rela��o �s caracter�sticas das ferramentas escolhidas e a partir da an�lise dos dados coletados ser� poss�vel concluir algo sobre o estado da pr�tica da �rea. As m�tricas levar�o em conta as diversas fases do uso das tecnologias como a instala��o e o uso da grade em si, como tamb�m custos de manuten��o e prepara��o de aplica��es para execu��o na grade, por exemplo.

Dessa forma, pode se dizer que a metodologia destes trabalho se divide nas seguintes etapas:

\begin{itemize}
  \item Formula��o do problema e dos procedimentos da pesquisa:
\begin{itemize}
  \item Defini��o do problema;
  \item Identifica��o das ferramentas a serem analisadas;
\end{itemize}
  \item Planejamento das m�tricas avaliadas:
\begin{itemize}
  \item Determinar o que deve ser estudado nas ferramentas;
  \item Identificar aspectos comuns nas ferramentas; 
\end{itemize}
  \item Determina��o os cen�rios de experimenta��o:
\begin{itemize}
  \item Determinar a sele��o de amostras;
  \item Verificar e validar a sele��o de amostras no tocante a representar a popula��o; 
\end{itemize}
  \item Realiza��o os experimentos e coleta de dados;
  \item Desenvolvimento de um plano de an�lise da amostra:
\begin{itemize}
  \item Filtrar dados relevantes;
\end{itemize}
  \item Processar e analisar os dados obtidos.
\end{itemize}

% Metodologia Cient�fica
% 
% M�todos Qualitativos
%     Se ocupam de vari�veis que n�o podem ser medidas, apenas observadas.
% 
%     * Pesquisa observacional: observar o ambiente mas n�o modific�-lo
%           o perspectiva filos�fica:
%                 + positivistas: existem vari�veis objetivas. Tenta falar em teorias, provar ou desprov�-las.
%                       # Descritiva: Descreve de forma objetiva e direta eventos e fatos de interesse.
%                       # Explorat�ria: prop�e novas teorias, observa��es e m�tricas.
%                 + interpretativistas: n�o existem vari�veis objetivas. Tudo depende da interpreta��o do observador.
%                 + cr�tica: entende o mundo como uma constru��o hist�rica e social de rela��es de poder e domina��o. Vai atr�s dessas rela��es.
%           o Estilo:
%                 + Estudo de Caso: intera��o do pesquisador com o sujeito � semi-formal.
%                 + Etnografia: o pesquisador vive e trabalha com o sujeito.
%           o T�cnicas: o problema � o rigor. Garantir confiabilidade, validade (controlar a subjetividade ou vi�s do pesquisador) e generabilidade
%                 + amostragem fundamentada em teoria ou direcionada (purposive or theoretical sampling): sele��o das amostras n�o � aleat�ria, busca casos extremos
%                 + separa��o de observa��o e teoriza��o: coleta de dados e teoriza��o independentes
%                 + teoria fundamentada em dados (grounded theory): extrair dos dados qualitativos as teorias que os explicam
%                 + triangula��o: pelo menos duas fontes/formas para cada dado e an�lise da pesquisa. Mais de uma t�cnica de coleta de dados, ou mais de um pesquisador (codifica��o m�ltipla)
%                 + parceiro neutro: utilizar um pesquisador experiente n�o envolvido diretamente na pesquisa
%                 + valida��o pelos sujeitos: mostrar dados coletado e/ou a an�lise dos mesmos para os sujeitos da pesquisa.
%     * Pesquisa-a��o: objetivo central da pesquisa � modificar o ambiente atrav�s de implanta��o de um novo sistema, por exemplo.


\newpage
\section{Atividades Planejadas}
\label{atividades}

As seguintes atividades ser�o realizadas:
\begin{description}
  \item[Levantamento bibliogr�fico:] Apesar da natureza pr�tica da avalia��o qualitativa, se faz necess�rio uma investiga��o bibliogr�fica a fim de se identificar as mais importantes grades computacionais de desktop usadas atualmente;
  \item[Elabora��o das m�tricas de interesse para a avalia��o:] Essa atividade visa identificar os pontos nos quais as tecnologias ser�o comparadas e contrastadas;
  \item[Projeto de experimentos:] Planejar os experimentos � necess�rio para se cobrir os detalhes de interesse das m�tricas identificadas;
  \item[Prepara��o do ambiente experimental:] A avalia��o pr�tica reques instalar cada uma das ferramentas identificadas;
  \item[Realiza��o dos experimentos:] Com a finalidade de coletar observar as m�tricas e coletar dados sobre elas, quando poss�vel;
  \item[Apresenta��o de semin�rio com resultados:] Para divulgar na universidade para laborat�rios interessados a pesquisa realizada e os dados coletados;
  \item[Escrita de artigo para poss�vel publica��o:] Para divulgar para a comunidade cient�fica;
  \item[Escrita do relat�rio de est�gio:] Documentar a pesquisa realizada;
  \item[Prepara��o da apresenta��o de defesa do est�gio:] Preparar apresenta��o da pesquisa e conclus�es obtidas. 
\end{description}


A estimativa de horas a se realizar as atividades pode ser vista na tabela~\ref{table:atividades}.

\vspace{1cm}

\begin{table}[h]
\centering
\begin{tabular}{|c|c|c|}
    \hline
& Atividades & Horas estimadas\\
	\hline
A1 & Levantamento bibliogr�fico & 30\\
	\hline
A2 & Elabora��o das m�tricas de interesse para a avalia��o & 30\\
	\hline
A3 & Projeto de experimentos a serem realizados usando as grades de desktops & 20\\
	\hline
A4 & Prepara��o do ambiente experimental & 20\\
	\hline
A5 & Realiza��o de experimentos e an�lise de resultados & 100\\
	\hline
A6 & Apresenta��o de semin�rio com resultados & 20\\
	\hline
A7 & Escrita de artigo para poss�vel publica��o em uma confer�ncia nacional & 50\\
	\hline
A8 & Escrita do relat�rio de est�gio & 50\\
	\hline
A9 & Prepara��o da apresenta��o de defesa do est�gio & 10\\
	\hline
 & Total & 330\\
	\hline
\end{tabular}
\caption{Atividades e horas estimadas}
\label{table:atividades}
\end{table}


\newpage
\section{Resultados Esperados}
\label{resultados}

Como resultado principal do est�gio, espera-se obter uma tabela que contemple as grades avaliadas e as compare, com base nas m�tricas identificadas ao longo do estudo. Atrav�s de tal tabela, ser� poss�vel identificar claramente pontos positivos e negativos de cada tecnologia.



\section{Cronograma}
\label{cronograma}

Pretende-se realizar as atividades enumeradas na Se��o~\ref{atividades} segundo o cronograma representado na Tabela~\ref{table:cronograma}: 

\begin{table}[h]
\centering
\begin{tabular}{|c|c|c|c|c|c|c|c|c|c|c|c|c|c|c|c|c|}
    \hline
    & \multicolumn{2}{|c|}{Mar�o} & \multicolumn{4}{|c|}{Abril} & \multicolumn{4}{|c|}{Maio} & \multicolumn{4}{|c|}{Junho} & \multicolumn{2}{|c|}{Julho}\\
    \cline{2-17}
& & & & & & & & & & & & & & & & \\
	\hline
A1 & \cellcolor{black} & \cellcolor{black} & & & & & & & & & & & & & & \\
	\hline
A2 & & \cellcolor{black} & \cellcolor{black} & & & & & & & & & & & & & \\
	\hline
A3 & & & & \cellcolor{black} & & & & & & & & & & & & \\
	\hline
A4 & & & & & \cellcolor{black} & & & & & & & & & & & \\
	\hline
A5 & & & & & & \cellcolor{black} & \cellcolor{black} & \cellcolor{black} & \cellcolor{black} & \cellcolor{black} & & & & & & \\
    \hline
A6 & & & & & & & & & & & \cellcolor{black} & & & & & \\
    \hline
A7 & & & & & \cellcolor{black} & \cellcolor{black} & \cellcolor{black} & \cellcolor{black} & \cellcolor{black} & \cellcolor{black} & \cellcolor{black} & \cellcolor{black} & \cellcolor{black} & \cellcolor{black} & & \\
    \hline
A8 & & & & & & & \cellcolor{black} & \cellcolor{black} & \cellcolor{black} & \cellcolor{black} & \cellcolor{black} & \cellcolor{black} & \cellcolor{black} & \cellcolor{black} & \cellcolor{black}& \cellcolor{black} \\
    \hline
A9 & & & & & & & & & & & & & & & \cellcolor{black} & \cellcolor{black} \\
	\hline
\multicolumn{1}{c}{\rule{1.0cm}{0cm}} &
\multicolumn{1}{c}{\rule{0.4cm}{0cm}} &
\multicolumn{1}{c}{\rule{0.4cm}{0cm}} &
\multicolumn{1}{c}{\rule{0.4cm}{0cm}} &
\multicolumn{1}{c}{\rule{0.4cm}{0cm}} &
\multicolumn{1}{c}{\rule{0.4cm}{0cm}} &
\multicolumn{1}{c}{\rule{0.4cm}{0cm}} &
\multicolumn{1}{c}{\rule{0.4cm}{0cm}} &
\multicolumn{1}{c}{\rule{0.4cm}{0cm}} &
\multicolumn{1}{c}{\rule{0.4cm}{0cm}} &
\multicolumn{1}{c}{\rule{0.4cm}{0cm}} &
\multicolumn{1}{c}{\rule{0.4cm}{0cm}} &
\multicolumn{1}{c}{\rule{0.4cm}{0cm}} &
\multicolumn{1}{c}{\rule{0.4cm}{0cm}} &
\multicolumn{1}{c}{\rule{0.4cm}{0cm}} &
\multicolumn{1}{c}{\rule{0.4cm}{0cm}} &
\multicolumn{1}{c}{\rule{0.4cm}{0cm}} \\
\end{tabular}
\caption{Cronograma}
\label{table:cronograma}
\end{table}


\newpage
\bibliography{plano}

\newpage
\section{Aprova��o}
\label{aprovacao}

Declaro para os devidos fins que aprovo o planejamento das atividades descritas neste documento como plano de est�gio do aluno Ricardo Ara�jo Santos, matr�cula 20511120.

\begin{center}
	\vspace{3cm}
	\rule{12cm}{0.1mm}\\
	\textbf{Marcus Williams Aquino de Carvalho}\\
	Supervisor T�cnico

	\vspace{3cm}
	\rule{12cm}{0.1mm}\\
	\textbf{Raquel Vigolvino Lopes}\\
	Supervisora Acad�mica

	\vspace{3cm}
	\rule{12cm}{0.1mm}\\
	\textbf{Joseana Mac�do Fechine}\\
	Coordenadora da Disciplina Est�gio Integrado

\end{center}

\end{document}
