\section{Resumo do Problema}
\label{problema}

A pesquisa cient�fica em algumas �reas tem demandado cada vez mais poder computacional, seja para a realiza��o de simula��es como para a execu��o de experimentos. Uma sa�da natural � a aquisi��o de supercomputadores ou m�quinas dedicadas (clusters) exclusivamente � computa��o paralela, o que nem sempre � poss�vel dado seu elevado custo. As grades computacionais \cite{berman, kesselman} surgiram com a id�ia de resolver tais problemas computacionais de maneira eficiente e com baixo custo, atrav�s do compartilhamento de recursos.

Uma forma comum de compartilhamento de recursos faz uso de poder computacional ocioso, formando o que se chama no contexto desse trabalho de grades computacionais oportunistas. Exemplos de grades oportunistas s�o: Condor~\cite{condor}, BOINC~\cite{boinc}, XtremWeb~\cite{xtremweb} e OurGrid~\cite{ourgrid}.

Muito tem sido publicado em confer�ncias e jornais sobre essas ferramentas. No entanto, n�o s�o conhecidos estudos emp�ricos que visem uma caracteriza��o do uso das ferramentas disponibilizadas, impossibilitando identificar o estado-da-pr�tica desta �rea. Acredita-se que um estudo pr�tico, em contrapartida aos estudos te�ricos conhecidos, � de grande import�ncia tanto para identifica��o de lacunas quanto para sele��o do melhor servi,o em diferentes contextos.
