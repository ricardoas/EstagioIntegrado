\pagestyle{empty}

\begin{center}
{\textbf{\Large \textsc{Universidade Federal de Campina Grande}}}
\end{center}

\begin{center}
\textbf{{\Large \textsc{Centro de Engenharia El�trica e
Inform�tica}}}
\end{center}

\begin{center}
\textbf{{\Large \textsc{Unidade Acad�mica de Sistemas e
Computa��o}}}
\end{center}

\begin{center}
{\large \textsc{\textbf{Gradua��o em Ci�ncia da Computa��o}}}
\end{center}


~\\


\begin{center}
\textsc{Relat�rio de Est�gio}\\
{\Large \textsc{\textbf{An�lise Comparativa do Uso de Tecnologias de Grades Computacionais de Desktop}}}
\end{center}
~\\

\begin{center}
\large{\textsc{Ricardo Ara�jo Santos}}\\
\textsc{Estagi�rio}\\
\end{center}
~
% \begin{quote}
% \small{Monografia submetida � Coordena��o do Curso de Ci�ncia da
% Computa��o da Universidade Federal de Campina Grande - Campus I como
% resultado da disciplina Est�gio Integrado.}
% \end{quote}
% ~\\
% 
% \begin{center}
% \textsc{�rea de Concentra��o: Ci�ncia da Computa��o }\\
% \textsc{Linha de Pesquisa: Sistemas Distribu�dos - Sistemas Peer-to-Peer - Grades Computacionais}\\
% \end{center}
% ~\\

\begin{center}
\textsc{Raquel Vigolvino Lopes}\\
\textsc{Orientadora Acad�mica}\\
\end{center}
~

\begin{center}
\textsc{Marcus Williams Aquino de Carvalho}\\
\textsc{Supervisor T�cnico}\\
\end{center}
~

\begin{center}
{\small \textsc{Campina Grande, Para�ba, Brasil}}\\
{\small \textsc{$\copyright$ Ricardo A. Santos, Julho de 2009}}
\end{center}

\newpage
\cleardoublepage