\chapter{Ambiente de Est�gio}
\label{ambiente}

O est�gio foi desenvolvido no Laborat�rio de Sistemas Distribu�dos (LSD) do Departamento de Sistemas e Computa��o (DSC) da Universidade Federal de Campina Grande.

O laborat�rio foi criado em 1996, como forma de aglutinar pesquisadores e alunos do DSC e de outros departamentos em torno de projetos na �rea de Sistemas Distribu�dos. A\-tu\-al\-men\-te o LSD � coordenado pelo professor Francisco Brasileiro. As pesquisas do LSD est�o concentradas em Grades Computacionais, Sistemas \textit{Peer-to-Peer}, \textit{Cloud Computing}, Toler�ncia a Falhas, Desenvolvimento de Software Distribu�do e Aplica��es Industriais.

\section{Estrutura F�sica}
\label{estrutura}

O LSD est� instalado num pr�dio com $550m^2$ de �rea e conta com a colabora��o de dezenas de alunos desenvolvendo trabalhos de doutorado, mestrado e inicia��o cient�fica, al�m de v�rios pesquisadores. Os diferentes projetos em execu��o est�o distribu�dos em 8 (oito) salas climatizadas, cada uma contanto com quadro branco, pinc�is e um acervo bibliogr�fico de diversos temas em computa��o e engenharia. Cada pessoa possui um posto de trabalho individual, com m�quinas conectadas � Internet via POP-PB da RNP\nomenclature{RNP}{Rede Nacional de Pesquisa}.

\vspace{1cm}
\noindent
\textbf{Endere�o}: Universidade Federal de Campina Grande\newline
Departamento de Sistemas e Computa��o\newline
Laborat�rio de Sistemas Distribu�dos\newline
Av. Apr��gio Veloso, 882 - Bloco CO\newline
Bodocong�, CEP 58109-970 Campina Grande - PB, Brasil\newline
Fone: +55 83 3310 1365\newline
Fax: +55 83 3310 1498

\section{Rucursos Utilizados}

\subsection{Hardware}

Foram disponibilizados quatro computadores de configura��es distintas. 

\subsection{Software}

Os recursos de software utilizados foram os seguintes:

\begin{itemize}
  \item Eclipse: Como ambiente de desenvolvimento da aplica��o Java usada para testar as grades computacionais. Al�m disso, o Eclipse em conjunto com o plugin Texlipse foi usado como editor de texto \LaTeX;
  \item VServer: Como ambiente de m�quinas virtuais.
\end{itemize}
